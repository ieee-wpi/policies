\chapter{Voting Procedure}\label{Bylaw:Voting}

\section{Voting Membership}\label{Bylaw:Voting:Membership}
	The \gls{VotingMembership} is the set of people who are eligible vote in a certain ballot. 
	\begin{itemize}
		\item{\textbf{Passing a Motion}: When balloting for a motion, the \gls{VotingMembership} shall be the \gls{OfficerBoard}.}
		\item{\textbf{Electing an Executive}: When balloting to elect an \gls{Executive}, the \gls{VotingMembership} shall be the \glspl{ActiveMember}.}
		\item{\textbf{Appointing an Officer}: When balloting to appoint an \gls{Officer}, the \gls{VotingMembership} shall be the \gls{OfficerBoard}.}
	\end{itemize}

\section{Passing a Motion}\label{Bylaw:Voting:Motion}
	When balloting for a motion, each member of the \gls{VotingMembership} is entitled to a single vote  either \textit{in-favor-of} or \textit{against} the vote. The motion will pass provided that two-thirds of the \gls{VotingMembership} votes \textit{in-favor-of} the motion. At the digression of the moderator, the votes may be cast either publicly or anonymously. If the votes are cast publicly, the moderator shall not be entitled to a vote.

\section{Electing an Executive}\label{Bylaw:Voting:Electing}
	Balloting to elect an \gls{Executive} shall occur during an \gls{GeneralBodyMeeting} and shall proceed as follows:
	\begin{enumerate}
		\item{If there are no candidates for the position, the position shall be left vacant.}
		\item{If there are two candidates for the position, each member of the \gls{VotingMembership} shall be entitled to an anonymous vote for one candidate. The candidate with the largest number of votes shall be elected to the \gls{Executive} position. In the case of a tie, the moderator shall anonymously selected the candidate to be elected to the \gls{Executive} position. \label{Bylaw:Voting:Electing:StepTwoCandidate}}
		\item{If there are more than two candidates for the position, each member of the \gls{VotingMembership} shall be entitled to an anonymous vote for two candidates. The ballot shall proceed (return to step \ref{Bylaw:Voting:Electing:StepTwoCandidate}) with the two candidates with the largest number of votes. In the case of a tie, the ballot shall proceed with the two candidates anonymously selected by the moderator.}
	\end{enumerate}

\section{Appointing an Officer}\label{Bylaw:Voting:Appointing}
	Balloting to appoint an \gls{Officer} shall occur during an \gls{GeneralBodyMeeting} and shall proceed as follows:
	\begin{enumerate}
		\item{If there are no candidates for the position, the position shall be left vacant.}
		\item{If there is one candidate for the position, each member of the \gls{VotingMembership} shall be entitled to an anonymous vote either \textit{in-favor-of} or \textit{against} candidate. The candidate shall be appointed to the \gls{Officer} position provided that two-thirds of the \gls{VotingMembership} votes \textit{in-favor-of} the candidate. Otherwise the position shall be left vacant.\label{Bylaw:Voting:Appointing:StepOneCandidate}}
		\item{If there are two candidates for the position, each member of the \gls{VotingMembership} shall be entitled to an anonymous vote for one candidate. The ballot shall proceed (return to step \ref{Bylaw:Voting:Appointing:StepOneCandidate}) with the candidate with the largest number of votes. In the case of a tie, the ballot shall proceed with the candidate anonymously selected by the moderator. \label{Bylaw:Voting:Appointing:StepTwoCandidate}}
		\item{If there are more than two candidates for the position, each member of the \gls{VotingMembership} shall be entitled to an anonymous vote for two candidates. The ballot shall proceed (return to step \ref{Bylaw:Voting:Appointing:StepTwoCandidate}) with the two candidates with the largest number of votes. In the case of a tie, the ballot shall proceed with the two candidates anonymously selected by the moderator.}
	\end{enumerate}
